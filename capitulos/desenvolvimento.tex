\chapter{Fundamentação Teórica}\label{cap:CnptDsng}

\section*{Processamento de Áudio}\label{sec:esc_freq}
As gravações de áudio se dão por meio de microfones e captadores de áudio, os quais realizam a conversão por meio de um transdutor, o qual converte a variação da pressão acústica em uma variação de tensão elétrica correspondente. Este sinal é convertido em pequenas amostras individuais espaçadas no tempo de maneira regular, constituindo a aproximação da forma de onda original.

Este processo é conhecido como conversão analógico-digital. O número de amostras retiradas da onda original no período de um segundo, é chamada frequência de amostragem, e quanto mais elevado o seu valor, mais fiel será a representação do sinal no domínio digital. De acordo com o teorema de Nyquist, o limite mínimo para a frequência de amostragem de qualquer sinal é o dobro da sua frequência original. A representação do sinal de áudio no domínio digital é apresentada como uma sequência de palavras onde o número de bits determina a resolução em amplitude do sinal.

No processo de reprodução de áudio, acontece o inverso da situação original, onde o sinal digital é enviado a um conversor digital-analógico, responsável pela reconstrução do sinal para que ele possa ser reproduzido em alto-falantes, caixas de som, ou qualquer aparelho que possa reproduzir sinais de áudio.

Dentro do domínio digital, o sinal de áudio pode ser tratado utilizando todas as técnicas de processamento, tais como as FFTs, DFTs filtros digitais, técnicas de janelamento e filtros digitais. Sendo assim, torna-se extremamente interessante e necessário realizar diversos tratamentos nestes sinais no domínio digital para reduzir as interferências externas, ruídos e desafinação, mantendo assim uma afinação constante em uma música. 

Uma das principais técnicas utilizadas para correção de afinação é a de pitch-shift, a qual consiste na mudança do tom de um sinal de áudio sem modificar o tamanho dele. A diferença de um deslocamento em frequência para o pitch-shift é justamente o fato de que num deslocamento em frequência, existe um deslocamento do espectro do som, enquanto um phase-shift dilata ou comprime o espectro de som.. O Pashe-Vocoder é uma técnica de pitch-shift muito utilizada por produtores musicais. 


\section*{STFT (Short Time Fourier Transform)}\label{sec:est_obs}
Uma das técnicas mais utilizadas em processamento digital de sinais é a Transformada Discreta de Fourier (DFT), a qual é uma variação da Transformada de Fourier para sinais discretos. A transformada de Fourier define a redução de uma função periódica a um somatório de senos e cossenos. Este procedimento matemático gera uma representação de um sinal originalmente no domínio do tempo, em uma representação no domínio da frequência.

\begin{figure}[h]
	\centering
	\includegraphics[width=.3\textwidth]{Fourier.png}
	\label{fig:Fourier}
	\caption{Transformada de Fourier para sinais contínuos}
	%\source{Própria}
\end{figure}

\begin{figure}[h]
	\centering
	\includegraphics[width=.5\textwidth]{DFT.png}
	\label{fig:DFT}
	\caption{Transformada Discreta de Fourier}
	%\source{Própria}
\end{figure}

As transformadas de Fourier se aplicam somente a sinais de funções estacionárias, onde o espectro de frequência é fixo e não variam com o tempo.

Os sinais de áudio gerados pela voz humana se encontram dentro do espectro de frequências entre 50 a 3400 Hz, e a sua principal característica é a de que a sua frequência não é constante no tempo, o que dá ao sinal da voz humana a característica da não-ergodicidade (seu sinal não mantém as propriedades estatísticas ao longo do tempo), sendo assim, a utilização da STFT (Short Time Fourier Transform) se torna bastante eficaz em sinais dessa natureza.

A STFT é um algoritmo desenvolvido com base na transformada discreta de Fourier, diferenciando-se pela inclusão de uma função de janelamento w(t). Sua principal aplicação é para funções cujo o espectro de frequência varia com o tempo.

\begin{figure}[h]
	\centering
	\includegraphics[width=.6\textwidth]{STFT.png}
	\label{fig:STFT}
	\caption{Transformada de Fourier de Tempo Curto}
	%\source{Própria}
\end{figure}

\begin{figure}[h]
	\centering
	\includegraphics[width=.7\textwidth]{STFT_GRAPHICS.png}
	\label{fig:STFT_GRAPHICS}
	\caption{Transformada de Fourier de Tempo Curto sendo aplicada em diversas frequências de um mesmo sinal}
	%\source{Própria}
\end{figure}
\newpage
 
O principal propósito da utilização de uma STFT é separar o sinal em pequenos intervalos que possam ser tratados individualmente, obtidos através da janela que está inserida na transformada. Desta maneira a modificação de frequência se dá de forma independente, sem a alteração de tempo e vice versa. 

\section*{Funções de Janelamento}
Para aplicações que consistem na amostragem de sinais, a amostragem, por ser finita, resulta em uma forma de onda truncada com características diferentes do sinal original, consequentemente a influência do vazamento espectral torna-se maior para uma situação como esta, gerando uma perda de informação do sinal original. 
Para reduzir os efeitos das imperfeições de amostragem, melhorando a qualidade da reconstrução do sinal é a aplicação de uma função de janelamento.

\begin{figure}[h]
	\centering
	\includegraphics[width=.7\textwidth]{Janelamento.png}
	\label{fig:Janelamento}
	\caption{Efeitos da função de janelamento em um sinal no espectro de frequência}
	%\source{Própria}
\end{figure}
 \newpage

A utilização de uma função de janelamento permite uma definição do período de observação do sinal, redução dos efeitos do vazamento espectral e a separação do sinal de pequena amplitude com frequências muito próximas. A aplicação de uma função de janelamento no tempo, consiste na multiplicação da função original pela função, o que equivale a uma convolução no domínio da frequência.
Existem diversas funções de janelamento, as quais possuem diferentes características e aplicações dependendo principalmente dos parâmetros desejados do sinal original.

\begin{itemize}
	\item Retangular
	\item Hanning
	\item Hamming
	\item Blackman
	\item Kaiser-Bessel
\end{itemize}
 
\section*{Janela Hanning}

Dentre as funções de janelamento existentes, a função Hanning é a mais comumente utilizada na produção musical. O formato desta janela é similar ao de meio ciclo de uma onda cossenoidal. Suas características de baixo vazamento espectral e  formato de onda bem similar ao formato cossenóide,. Torna-se recomendável utilizar portanto a janela para análises de sinais com transientes maiores que de duração da própria janela.

\begin{equation}
w[n] = 0.5-0.5*\cos(\dfrac{2\pi*n}{N}), n = 0, 1, 2,..., N-1 
\end{equation}

\begin{figure}[h]
	\centering
	\includegraphics[width=.7\textwidth]{Janela_Hanning.png}
	\label{fig:Janela Hanning}
	\caption{Função da janela Hanning}
	%\source{Própria}
\end{figure}

\begin{figure}[h]
	\centering
	\includegraphics[width=.5\textwidth]{Grafico_Hanning.png}
	\label{fig:Gráfico Hanning}
	\caption{Função da janela Hanning}
	%\source{Própria}
\end{figure}






\chapter{Resultados}
\section{Dados de entrada}
\subsection{Dados Gerais}
A altura escolhida para as torres foi de 0 e 9 para manter a linha de visada direta em paralelo com o chao e assim facilitando os cálculos e diminuindo os erros, os outros dados estão mostrados na tabela abaixo.
	\begin{table}[h]
		\centering
		\begin{tabular}{|
				>{\columncolor[HTML]{DAE8FC}}c |l|}
			\hline
			Distância Total (m) & 0,96713                  \\ \hline
			$\lambda$           & 0,001086957 x  $10^{-6}$ \\ \hline
			Altura das torres   & 0/9                      \\ \hline
		\end{tabular}
	\end{table}
\subsection{Dados dos Obstáculos}

\begin{table}[h]
	\centering
	\begin{tabular}{|
			>{\columncolor[HTML]{DAE8FC}}c |l|}
		\hline
		X                                                   & 325     \\ \hline
		Y                                                   & 7       \\ \hline
		$d_1$                                               & 0,475   \\ \hline
		\multicolumn{1}{|l|}{\cellcolor[HTML]{DAE8FC}$d_2$} & 0,49213 \\ \hline
	\end{tabular}
\end{table}
			
\section{Memorial de Cálculo}
\subsection{Frequência}
Não foi necessário calcular a frequência, o valor utilizado foi de 920MHz e sua escolha está justificada em \ref{sec:esc_freq}.

\subsection{Raio da Parábola}
Com os valores de $X$ e $Y$ do obstáculo o valor encontrado para o raio da parábola foi de:
\begin{equation}
	r_{parabola}= 1,886160714
\end{equation}

\subsection{Atenuação do obstáculo}
Com os valores de $d_1$ e $d_2$ o $\alpha$ encontrado teve valor de :
\begin{equation}
	\alpha = 0,6703932617
\end{equation}

O parâmetro $H_c$ foi encontrado por meio dos dados fornecidos, como mostrado na figura a seguir:

O parâmetro $r_f$ foi cálculado por meio da equação de $Fresnel$, com o valor dos parâmetros a seguir, foi encontrado no gráfico o valor da atenuação do obstáculo.

\begin{table}[h]
	\centering
	\begin{tabular}{|
			>{\columncolor[HTML]{DAE8FC}}c |l|}
		\hline
		$r_f$                                                         & 8,866    \\ \hline
		$H_c$                                                         & 5        \\ \hline
		$\dfrac{H_c}{r_f}$                                            & 0,563    \\ \hline
		\multicolumn{1}{|l|}{\cellcolor[HTML]{DAE8FC}$L_{obstaculo}$} & $22,5dB$ \\ \hline
	\end{tabular}
\end{table}

\subsection{Atenuação no Espaço livre}
Com os valores de $d_1$, $d_2$ e $f$ conhecidos o valor encontrado para $L$ foi:
\begin{equation}
L = 91,43dB
\end{equation}

\subsection{Atenuação total}
A atenuação total consiste nas soma das atenuação encontradas, logo:
\begin{equation}
L_{tot} = L + L_{obstaculo} = 91,43 + 22,5 = 113,9dB 
\end{equation} 

\section{Escolha do tranmissor e da antena}
 A escolha dos módulos e da antena se deu baseado na frequência escolhida para os cálculos e na versatilidade de cada dispositivo. 
 
 O modelo das antenas foi o \textit{Yagi(AirMax Antenna 900Mhz)},devido a sua faixa de trabalho e alta potência.
 
 O modelo do transmissor foi o \textit{Ubiquiti Networks(Rocket M9)} pot ser recomendado para trabalhar em conjunto com o modelo de antena \textit{Yagi}.

\section{Receptor}
Escolha do receptor é encontrado à partir das potências das antenas, do módulo tranmissor e das perdas durante a transmissão. Sua potência foi calculada da seguinte forma:
\begin{equation}
	P_{receptor} = P_{antena_tx} + P_{antena_rx} + P_{transmissor} - L_{tot}
\end{equation}

A mesma antena utilizada para transmissão é utilizada para recepção, os dados da sua potência estão disponíveis nos seu \textit{datasheet} onde $P_{antena_{tx}} = P_{antena_{rx}} = 19dBi$. A potência do transmissõr também está disponível no datasheet, onde $P_{transmissor} = 28dBm$. O valor encontrado para potência do receptor foi de $	P_{receptor} = -47.9dBi$

Com o valor de $-47.9dBi$, o módulo \textit{Ubiquiti Networks(Rocket M9)} também poderá ser utilizado para recepção, tornando o sistema mais simplificado já que ambos receptores e transmissores estarão utilizando antenas recomendadas no datasheet.

\newpage
\begin{center}
	\Large \textbf{Variáveis do projeto}
\end{center}
Todos os valores utilizados e calculados estão registrados na figura a seguir:
\begin{figure}[h]
	\centering
	\includegraphics[width=1\textwidth]{resultados.jpeg}
	\label{fig:result}
	\caption{Variáveis do projeto}
	%\source{Própria}
\end{figure} 

