% ----------------------------------------------------------
% Introdução 
% Capítulo sem numeração, mas presente no Sumário
% ----------------------------------------------------------

\chapter[Introdução]{Introdução}
\addcontentsline{toc}{chapter}{Introdução}

Este relatório descreve os procedimentos e códigos utilizados para a criação de um sistema de processamento de áudio para correção de tons musicais.

\section*{Teoria Musical}
Na música a notação utilizada, chamada diastemática, os sons são representados graficamente, de maneira que seja possível mensurar os intervalos de frequências, o que indica diferentes notas musicais. 

Para representação de sons mais longos ou curtos surge então a notação da grandeza tempo, que diferentemente da convenção comum, não possui um valor fixo, portanto, cria-se uma referência com relação à duração da notas semibreve, sendo elas a mínima, semínima, colcheia, semicolcheia, fusa, semifusa.

Outro aspecto extremamente importante da teoria musical é a frequência, grandeza qual classifica o quão graves ou agudos são os sons. Sons que apresentam maiores frequências são mais agudos que os de frequências mais baixas 

Diferentes instrumentos são capazes de gerar sons diferentes mesmo que as frequências fundamentais sejam idênticas, o que em outras palavras significa dizer que um Ré gerado por um instrumento de corda não tem o mesmo som de um Ré gerado por um instrumento de sopro ou da voz humana, Este fenômeno, que é chamado de timbre, se dá principalmente pela geometria do instrumento, que define quais serão as outras componentes senoidais adicionadas à fundamental.[1]

A evolução da música faz com que o tratamento do sinal se torne uma questão cada vez mais explorada. A busca de um som refinado e que soe bem fomentou o desenvolvimento de  diversas técnicas para manipulação de sinais voltados para criação de novas experiências sonoras. 
    
O intercâmbio cultural gerado pelo crescimento da internet, fez com que o nível de habilidade exigido dos artistas aumentasse, assim sendo necessário a exploração de novas características para o músico. Ferramentas para geração de sons digitais como os sintetizadores, utilizam as propriedades dos sinais para criar sons não encontrados normalmente na natureza.
    
O auto-tune é uma ferramenta conhecida por “consertar” a voz dos cantores, tendo a função de corrigir os tons de voz do cantor para uma escala específica, deixando a voz afinada. O nome pertence ao plugin criado em 1997 pela Antares Audio Technologies. Desde então é utilizados em diversas músicas e de diferentes formas.


\section*{Proposta}\label{sec:motivacao}

Projetar um algoritmo capaz de realizar a afinação de um som emitido pela voz humana para uma frequência selecionada através das técnicas de processamento como transformadas e mudanças de tom. Para a realização dos algoritmos foram tomadas as seguintes decisões:

\begin{itemize}
\item Aplicação dos conceitos de Processamento digital de Sinais;
\item Identificação das componentes de frequência da voz;
\item Realizar a troca de tom através das técnicas de Pitch Shfit
\end{itemize}

